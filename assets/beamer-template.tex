% =============================================================================
% 技术调研报告 Beamer 模板
% 使用 ctex 宏包支持中文,适合技术演示
% =============================================================================

\documentclass[aspectratio=169,11pt]{beamer}

% -----------------------------------------------------------------------------
% 基础宏包
% -----------------------------------------------------------------------------
\usepackage[UTF8,fontset=windows]{ctex}          % 中文支持(Windows字体)
\usepackage{geometry}
\usepackage{graphicx}                            % 图片支持
\usepackage{xcolor}                              % 颜色支持
\usepackage{tikz}                                % 绘图
\usepackage{pgfplots}                            % 图表
\usepackage{booktabs}                            % 专业表格
\usepackage{multirow}                            % 表格多行合并
\usepackage{array}                               % 表格增强
\usepackage{colortbl}                            % 表格颜色
\usepackage{enumitem}                            % 列表增强
\usepackage{fontawesome5}                        % 图标
\usepackage{hyperref}                            % 超链接

% TikZ库
\usetikzlibrary{shapes.geometric, arrows.meta, positioning, calc, decorations.pathreplacing}
\pgfplotsset{compat=1.18}

% -----------------------------------------------------------------------------
% 主题和颜色配置
% -----------------------------------------------------------------------------
\usetheme{Madrid}
\usecolortheme{default}

% 自定义颜色 - 科技感蓝色调
\definecolor{primarycolor}{RGB}{41, 98, 255}      % 主色:科技蓝
\definecolor{secondarycolor}{RGB}{0, 150, 136}    % 辅色:青绿色
\definecolor{accentcolor}{RGB}{255, 111, 97}      % 强调色:珊瑚橙
\definecolor{darkbg}{RGB}{30, 40, 60}             % 深色背景
\definecolor{lightgray}{RGB}{245, 245, 245}       % 浅灰背景
\definecolor{textgray}{RGB}{80, 80, 80}           % 文字灰

% 设置主题颜色
\setbeamercolor{palette primary}{bg=primarycolor,fg=white}
\setbeamercolor{palette secondary}{bg=secondarycolor,fg=white}
\setbeamercolor{palette tertiary}{bg=darkbg,fg=white}
\setbeamercolor{structure}{fg=primarycolor}
\setbeamercolor{title}{fg=white,bg=primarycolor}
\setbeamercolor{frametitle}{fg=white,bg=primarycolor}
\setbeamercolor{block title}{bg=primarycolor,fg=white}
\setbeamercolor{block body}{bg=lightgray,fg=black}
\setbeamercolor{block title alerted}{bg=accentcolor,fg=white}
\setbeamercolor{block body alerted}{bg=lightgray,fg=black}
\setbeamercolor{item}{fg=primarycolor}
\setbeamercolor{subitem}{fg=secondarycolor}
\setbeamercolor{itemize item}{fg=primarycolor}
\setbeamercolor{itemize subitem}{fg=secondarycolor}
\setbeamercolor{enumerate item}{fg=primarycolor}
\setbeamercolor{section in toc}{fg=primarycolor}
\setbeamercolor{subsection in toc}{fg=secondarycolor}

% 设置字体
\setbeamerfont{title}{size=\LARGE,series=\bfseries}
\setbeamerfont{frametitle}{size=\Large,series=\bfseries}
\setbeamerfont{framesubtitle}{size=\normalsize}

% 取消导航符号
\setbeamertemplate{navigation symbols}{}

% 添加页码
\setbeamertemplate{footline}{
    \leavevmode%
    \hbox{%
        \begin{beamercolorbox}[wd=.333333\paperwidth,ht=2.25ex,dp=1ex,center]{author in head/foot}%
            \usebeamerfont{author in head/foot}\insertshortauthor
        \end{beamercolorbox}%
        \begin{beamercolorbox}[wd=.333333\paperwidth,ht=2.25ex,dp=1ex,center]{title in head/foot}%
            \usebeamerfont{title in head/foot}\insertshorttitle
        \end{beamercolorbox}%
        \begin{beamercolorbox}[wd=.333333\paperwidth,ht=2.25ex,dp=1ex,right]{date in head/foot}%
            \usebeamerfont{date in head/foot}\insertshortdate{}\hspace*{2em}
            \insertframenumber{} / \inserttotalframenumber\hspace*{2ex}
        \end{beamercolorbox}}%
    \vskip0pt%
}

% 项目符号样式
\setbeamertemplate{itemize item}{\scriptsize\raisebox{0.5ex}{\textbullet}}
\setbeamertemplate{itemize subitem}{\tiny\raisebox{0.5ex}{$\circ$}}
\setbeamertemplate{itemize subsubitem}{\tiny\raisebox{0.5ex}{\textendash}}

% -----------------------------------------------------------------------------
% 自定义命令
% -----------------------------------------------------------------------------

% 重点强调
\newcommand{\highlight}[1]{\textcolor{accentcolor}{\textbf{#1}}}

% 技术术语
\newcommand{\tech}[1]{\texttt{\textcolor{primarycolor}{#1}}}

% 数据高亮框
\newcommand{\databox}[2]{%
    \begin{tikzpicture}
        \node[fill=lightgray, rounded corners=5pt, inner sep=10pt, text width=#1] {#2};
    \end{tikzpicture}
}

% 时间线条目
\newcommand{\timelineitem}[3]{%
    \textbf{#1} \hfill \textcolor{textgray}{\small #2}\\
    #3\vspace{0.5em}
}

% -----------------------------------------------------------------------------
% 文档信息(请在主文件中覆盖这些定义)
% -----------------------------------------------------------------------------
\title[技术调研报告]{技术主题调研报告}
\subtitle{技术发展、市场现状与未来趋势分析}
\author{调研团队}
\institute{技术研究院}
\date{\today}

% =============================================================================
% 文档开始(以下为示例内容,实际使用时请替换)
% =============================================================================
\begin{document}

% -----------------------------------------------------------------------------
% 标题页
% -----------------------------------------------------------------------------
{
\setbeamertemplate{footline}{}
\begin{frame}
    \titlepage
    \begin{tikzpicture}[remember picture,overlay]
        \node[opacity=0.1] at ([xshift=4cm,yshift=-2cm]current page.north west) {
            \fontsize{120}{120}\selectfont\faIcon{microchip}
        };
    \end{tikzpicture}
\end{frame}
}

% -----------------------------------------------------------------------------
% 目录
% -----------------------------------------------------------------------------
\begin{frame}{目录}
    \tableofcontents
\end{frame}

% -----------------------------------------------------------------------------
% 第一章:研究背景
% -----------------------------------------------------------------------------
\section{研究背景}

\begin{frame}{研究背景}
    \begin{block}{研究动机}
        随着人工智能技术的快速发展,\highlight{编程智能体}(Coding Agents)正在成为软件开发领域的重要趋势。
    \end{block}
    
    \vspace{0.5em}
    \textbf{研究目标:}
    \begin{itemize}
        \item 梳理编程智能体技术的发展历程
        \item 分析当前主流产品与技术路线
        \item 识别关键技术挑战与突破
        \item 预测未来发展趋势
    \end{itemize}
\end{frame}

% -----------------------------------------------------------------------------
% 第二章:技术发展历史
% -----------------------------------------------------------------------------
\section{技术发展历史}

\begin{frame}{技术发展历程}
    \begin{columns}
        \column{0.6\textwidth}
        \begin{itemize}
            \item \textbf{2018-2020}:代码补全工具时代
            \item \textbf{2021-2022}:大模型代码生成
            \item \textbf{2023}:智能体概念兴起
            \item \textbf{2024}:端到端编程智能体
        \end{itemize}
        
        \column{0.4\textwidth}
        \begin{tikzpicture}[scale=0.7]
            \foreach \y/\year/\label in {3/2018/工具, 2/2021/生成, 1/2023/智能体, 0/2024/端到端} {
                \fill[primarycolor] (0,\y) circle (0.15);
                \node[right] at (0.3,\y) {\small \year};
                \node[right, text=textgray] at (1.2,\y) {\small \label};
                \ifnum\y>0
                    \draw[primarycolor, thick] (0,\y) -- (0,\the\numexpr\y-1\relax);
                \fi
            }
        \end{tikzpicture}
    \end{columns}
\end{frame}

% -----------------------------------------------------------------------------
% 第三章:市场现状与竞品分析
% -----------------------------------------------------------------------------
\section{市场现状与竞品分析}

\begin{frame}{主要参与者}
    \begin{columns}
        \column{0.5\textwidth}
        \textbf{国际厂商}
        \begin{itemize}
            \item GitHub Copilot (Microsoft)
            \item Cursor AI
            \item Amazon CodeWhisperer
            \item Google Duet AI
        \end{itemize}
        
        \column{0.5\textwidth}
        \textbf{国内厂商}
        \begin{itemize}
            \item 通义灵码 (阿里云)
            \item CodeGeeX (智谱AI)
            \item 文心快码 (百度)
            \item iFlyCode (科大讯飞)
        \end{itemize}
    \end{columns}
\end{frame}

\begin{frame}{竞品对比}
    \begin{table}
        \centering
        \small
        \rowcolors{2}{lightgray}{white}
        \begin{tabular}{lccc}
            \toprule
            \textbf{产品} & \textbf{代码生成} & \textbf{对话能力} & \textbf{价格} \\
            \midrule
            Copilot & \faIcon{star}\faIcon{star}\faIcon{star}\faIcon{star}\faIcon{star} & \faIcon{star}\faIcon{star}\faIcon{star} & \$10/月 \\
            Cursor & \faIcon{star}\faIcon{star}\faIcon{star}\faIcon{star} & \faIcon{star}\faIcon{star}\faIcon{star}\faIcon{star}\faIcon{star} & 免费/\$20 \\
            通义灵码 & \faIcon{star}\faIcon{star}\faIcon{star}\faIcon{star} & \faIcon{star}\faIcon{star}\faIcon{star} & 免费 \\
            \bottomrule
        \end{tabular}
    \end{table}
\end{frame}

% -----------------------------------------------------------------------------
% 第四章:关键技术解析
% -----------------------------------------------------------------------------
\section{关键技术解析}

\begin{frame}{技术架构}
    \begin{center}
    \begin{tikzpicture}[
        node distance=1.5cm,
        box/.style={rectangle, draw=primarycolor, fill=lightgray, rounded corners, 
                    minimum width=2.5cm, minimum height=0.8cm, align=center, font=\small},
        arrow/.style={-{Stealth[length=3mm]}, thick, primarycolor}
    ]
        \node[box] (input) {用户输入};
        \node[box, right=of input] (parse) {意图解析};
        \node[box, right=of parse] (plan) {任务规划};
        \node[box, below=of plan] (exec) {代码生成};
        \node[box, left=of exec] (verify) {验证测试};
        \node[box, left=of verify] (output) {结果输出};
        
        \draw[arrow] (input) -- (parse);
        \draw[arrow] (parse) -- (plan);
        \draw[arrow] (plan) -- (exec);
        \draw[arrow] (exec) -- (verify);
        \draw[arrow] (verify) -- (output);
    \end{tikzpicture}
    \end{center}
\end{frame}

\begin{frame}{核心技术挑战}
    \begin{alertblock}{主要挑战}
        \begin{itemize}
            \item \textbf{上下文理解}:大型代码库的语义理解
            \item \textbf{长程依赖}:跨文件的引用和修改
            \item \textbf{安全性}:生成代码的安全漏洞
            \item \textbf{可解释性}:生成逻辑的透明性
        \end{itemize}
    \end{alertblock}
\end{frame}

% -----------------------------------------------------------------------------
% 第五章:未来发展趋势
% -----------------------------------------------------------------------------
\section{未来发展趋势}

\begin{frame}{趋势预测}
    \begin{columns}
        \column{0.5\textwidth}
        \begin{block}{短期(1-2年)}
            \begin{itemize}
                \item 多模态输入支持
                \item 更精准的上下文感知
                \item 与开发工具深度集成
            \end{itemize}
        \end{block}
        
        \column{0.5\textwidth}
        \begin{block}{长期(3-5年)}
            \begin{itemize}
                \item 端到端软件生成
                \item 自主调试与优化
                \item 软件工程全流程覆盖
            \end{itemize}
        \end{block}
    \end{columns}
\end{frame}

% -----------------------------------------------------------------------------
% 第六章:总结
% -----------------------------------------------------------------------------
\section{总结与展望}

\begin{frame}{核心结论}
    \begin{enumerate}
        \item 编程智能体技术正处于\highlight{快速发展期}
        \item 大语言模型是核心驱动力
        \item 多智能体协作是未来方向
        \item 安全性和可控性是关键考量
    \end{enumerate}
    
    \vspace{1em}
    \begin{center}
        \databox{0.8\textwidth}{
            \centering
            \textbf{建议}:持续关注技术演进,适时引入工具提升效率
        }
    \end{center}
\end{frame}

% -----------------------------------------------------------------------------
% 结束页
% -----------------------------------------------------------------------------
{
\setbeamertemplate{footline}{}
\begin{frame}
    \begin{center}
        \Huge \textcolor{primarycolor}{感谢聆听}
        
        \vspace{1em}
        
        \Large Q \& A
        
        \vspace{2em}
        
        \normalsize
        \textcolor{textgray}{联系邮箱:research@example.com}
    \end{center}
    
    \begin{tikzpicture}[remember picture,overlay]
        \node[opacity=0.05] at (current page.center) {
            \fontsize{200}{200}\selectfont\faIcon{code}
        };
    \end{tikzpicture}
\end{frame}
}

\end{document}
